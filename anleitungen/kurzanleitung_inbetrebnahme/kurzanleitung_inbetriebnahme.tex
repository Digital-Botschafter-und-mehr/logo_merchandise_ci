% ungetestet
%
% Firmware: 0.6~stable20141018
% Gerät: TP-Link TL-WR841N

\documentclass{article}

\usepackage[utf8]{inputenc}	%Zeichencodierung Text
\usepackage[T1]{fontenc}
\newcommand{\changefont}[3]{
\fontfamily{#1} \fontseries{#2} \fontshape{#3} \selectfont}
\usepackage{textcomp}
\usepackage[ngerman]{babel}	%Wortdefinitionen
\usepackage[a5paper,%
 left=20mm,%
 right=20mm,%
 top=20mm,%
 bottom=20mm]{geometry}
\usepackage{fancyhdr}

\pagestyle{fancy}
\fancyhf{}
\rhead{freifunk-muenchen.de}
\lhead{Freifunk München}
\rfoot{Mehr Informationen zur Einrichtung unter www.freifunk-muenchen.de}

\renewcommand{\headrulewidth}{1pt}
\renewcommand{\footrulewidth}{1pt}

\renewcommand{\thefootnote}{\roman{footnote}}

\begin{document}

\changefont{phv}{m}{n}

\section*{Kurzanleitung zur Inbetriebnahme}

\begin{enumerate}
\item Schließe den Freifunk-Router mit dem mitgelieferten Netzteil an das Stromnetz an.

\item Stecke das mitgelieferte LAN-Kabel mit dem einen Ende in die \textbf{LAN}-Buchse deines Computers und mit dem anderen in die \textbf{WAN}-Buchse (blau) des Freifunk-Routers.

\item Rufe im Browser des Computers \textbf{192.168.1.1} auf. \label{webinterface}

\item Konfiguriere deinen Freifunk-Router. 
\begin{enumerate}  
  \item Optional können Geokoordinaten\footnote{Mit Hilfe des Werkzeugs \textit{Koordinaten beim nächsten Klicken anzeigen} der Knotenkarte auf freifunk-muenchen.de ermittelbar.} für den Standort des Freifunk-Routers und Kontaktdaten angegeben werden.

  \item Klicke unten rechts auf \glqq{}Fertig\grqq{}. \label{fertig}

  \textit{Die Konfigurationsseite des Freifunk-Routers ist nun nicht mehr erreichbar.} \footnote{Durch 3-5 sekündiges Drücken der Reset-Taste wird der Konfigurationsmodus des Routers wieder aktiviert. Anschließend kann wie in Schritt \ref{webinterface} fortgefahren werden.}
\end{enumerate}

\item (optional) Verbinde nun die \textbf{WAN}-Buchse (blau) des Freifunk-Routers mit einer \textbf{LAN}-Buchse Deines Heimrouters, der Verbindung zum Internet hat. \footnote{Der Freifunk-Router kann sich so mit Hilfe des Mesh-VPN über das Internet mit dem übrigen Freifunk München Intranet verbinden.}

\item Platziere den Freifunk-Router an einem Ort deiner Wahl. \label{platzieren}

\item Fertig!
\end{enumerate}

\end{document}
